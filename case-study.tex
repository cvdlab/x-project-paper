\section{Case study}\label{sec:case-study}
In this section we discuss the design and the implementation of a blog platform. 

For a blog platform the essential entities to model are: \texttt{Post} and \texttt{Tag}.

\begin{lstlisting}[language=json]
{
  "name": "Post",
  "properties": {
    "title": { "type": "string" },
    "posted": { "type": "date" },
    "content": { "type": "text" },
    "permalink": { "type": "string" }
  }, 
  "relations": {
    "tags": { "type": "has_many", "model": "Tag"}
  }
}
\end{lstlisting}

\begin{lstlisting}[language=json]
{
  "name": "Tag",
  "properties": {
    "name": { "type": "string" }
  }
}
\end{lstlisting}

These models results in the following HTTP RESTful API (automatically generated by Loopback server).

\begin{lstlisting}
GET|POST /api/Posts
GET|PUT|DELETE /api/Posts/:post_id
GET|POST /api/Tags
GET|PUT|DELETE /api/Tags/:tag_id
\end{lstlisting}

Since a snippet is worth a thousand words, in the following we present the pages of the app.
It's important to note how easily a page can be built without writing code but assembling elements. 

\paragraph{Admin part}

\begin{lstlisting}[language=HTML5]
<x-router>
  <x-route 
    route="/admin/:collection" 
    page="page-collection" />
  <x-route 
    route="/admin/:collection/:id"
    page="page-model-edit" />
</x-router>
\end{lstlisting}

\texttt{<page-collection>} shows models of a collection.

\begin{lstlisting}[language=HTML5]
<template name="page-collection">
  <api-collection-schema
    name="{collection}"
    schema="{schema}" />
  <api-collection-get 
    name="{collection}" where="{where}" 
    page="{page}" perpage="{perpage}"  
    items="{items}" count="{count}" />
  <api-collection-where 
    schema="{schema}"
    where="{where}" />
  <x-table 
    schema="{schema}"
    items="{items}" />
  <x-pager 
    count="{count}" perpage="{perpage}"
    current="{page}" />
</template>
\end{lstlisting}



\vspace{0.2cm}

\texttt{<page-model-edit>} shows the forms to update a model.

\begin{lstlisting}[language=HTML5]
<template name="page-model-edit">
  <api-collection-schema
    collection="{collection}"
    schema="{schema}" />
  <api-model-get 
    name="{collection}" model-id="{id}"
    model="{model}" />
  <x-form 
    schema="{schema}" 
    model="{model}" />
  <api-model-put 
    name="{collection}"
    model-id="{model_id}" />
</template>
\end{lstlisting}

\paragraph{User part}

\begin{lstlisting}[language=HTML5]
<x-router>
  <x-route route="/" page="page-posts" />
  <x-route route="posts/:id" page="page-post" />
</x-router>
\end{lstlisting}

\texttt{<page-posts>} show the list of posts.

\begin{lstlisting}[language=HTML5]
<template name="page-posts">
  <api-collection-get 
    name="Posts"
    perpage="10" page="{page}" 
    items="{posts}" count="{count}" />

  <template is="dom-repeat" items="{posts}">
    <li>{item.title} {item.posted}</li>
  </template>

  <x-pager 
    perpage="10" total="{count}" 
    current="{page}" />
</template>
\end{lstlisting}

\texttt{<page-post>} show a post. 
It is accessible via \texttt{/posts/:post\_id} route.
The \texttt{post\_id} parameter is picked from the url by the router and passed to the page.

\begin{lstlisting}[language=HTML5]
<template name="page-post">
  <api-model-get 
    name="Posts" model-id="{id}" 
    model="{post}" />

  <h1>{post.title}</h1>
  <h2>by {post.author}</h2>
  <h3>on {post.date}</h3>
  <div>{post.content}</div>
</template>
\end{lstlisting}

