\section{Case study}\label{sec:case-study}
In this section we discuss the design and the implementation of a blog platform. 

For a blog platform the essential entities to model are: \texttt{Post} and \texttt{Tag}.

\begin{lstlisting}[language=json]
{
  "name": "Post",
  "properties": {
    "title": { "type": "string" },
    "posted": { "type": "date" },
    "content": { "type": "text" },
    "permalink": { "type": "string" }
  }, 
  "relations": [{
    "name": "tags", 
    "type": "has_many", 
    "model": "Tag"
  }]
}
\end{lstlisting}

\begin{lstlisting}[language=json]
{
  "name": "Tag",
  "properties": {
    "name": { "type": "string" }
  }
}
\end{lstlisting}

These models results in the following HTTP RESTful API (automatically generated by Loopback server).

\begin{lstlisting}
GET|POST /api/Posts
GET|PUT|DELETE /api/Posts/:post_id
GET|POST /api/Tags
GET|PUT|DELETE /api/Tags/:tag_id
\end{lstlisting}

Since a snippet is worth a thousand words, in the following we present the pages of the app.
It's important to note how easily a page can be built without writing code but assembling elements. 

\vspace{0.2cm}

\texttt{index.html} imports the pages and set the router.
\texttt{<page-collection>} and \texttt{<page-model-edit>} are provided by the toolkit.

%\begin{lstlisting}[language=HTML5]
%<link rel="import" href="page-collection.html">
%<link rel="import" href="page-model-edit.html">
%<link rel="import" href="page-posts.html">
%<link rel="import" href="page-post.html">
%\end{lstlisting}

\begin{lstlisting}[language=HTML5]
<x-router>
  <x-route route="/admin/:collection" 
    page="page-collection"></x-route>
  <x-route route="/admin/:collection/:model_id"
    page="page-model-edit"></x-route>
  <x-route route="/" 
    page="page-posts"></x-route>
  <x-route route="posts/:id" 
    page="page-post"></x-route>
</x-router>
\end{lstlisting}

\texttt{<page-collection>} shows models of a collection.

\begin{lstlisting}[language=HTML5]
<template name="page-collection">
  <api-collection-get name="{{name}}" 
    where="{{filter}}" 
    page="{{page}}" perpage="{{perpage}}"  
    items="{{items}}" schema="{{schema}}"
    count="{{count}}">
  </api-collection-get>
  <api-filter schema="{{schema}}"
    filter="{{filter}}"></api-filter>
  <x-table schema="schema" editable
    items="{{items}}"></x-table>
  <x-pager count="{{count}}" perpage="{{perpage}}"
    current="{{page}}"></x-pager>
</template>
\end{lstlisting}

\vspace{0.2cm}

\texttt{<page-model-edit>} shows the forms to update a model.

\begin{lstlisting}[language=HTML5]
<template name="page-model-edit">
  <api-model-get name="{{collection}}"
    model-id="model_id"
    model="{{model}}" schema="{{schema}}">
  </api-model-get>
  <x-form schema="{{schema}}" 
    model="{{model}}"></x-form>
  <api-model-put name="{{collection}}"
    model-id="{{model_id}}"></api-model-put>
</template>
\end{lstlisting}

\vspace{0.2cm}

\texttt{<page-posts>} show the list of posts.

\begin{lstlisting}[language=HTML5]
<template name="page-posts">
  <api-collection-get name="Posts"
    page="{{page}}" perpage="10"
    collection="{{posts}}" count="{{count}}">
  </api-collection-get>
  <template is="dom-repeat" items="{{posts}}">
    <div>
      <h1>{{item.title}}</h1>
      <h3>{{item.posted}}</h3>
    <div>
  </template>
  <x-pager perpage="10" total="{{count}}" 
    current="{{page}}"></x-pager>
</template>
\end{lstlisting}

\texttt{<page-post>} show a post. 
It is accessible via \texttt{/posts/:post\_id} route.
The \texttt{post\_id} parameter is picked from the url by the router and passed to the page.

\begin{lstlisting}[language=HTML5]
<template name="page-post">
  <api-model-get name="Posts" 
    model-id="{{post_id}}" model="{{post}}">
  </api-model-get>
  <div>
    <h1>{{post.title}}</h1>
    <h2>by <span>{{post.author}}</span></h2>
    <h3>on <span>{{post.date}}</span></h3>
  </div>
  <div>{{post.content}}</div>
</template>
\end{lstlisting}

