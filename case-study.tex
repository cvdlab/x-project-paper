\section{Case study}
In this section we discuss the design and the implementation of a blog platform. 

\subsection{Models}

For a blog platform the essential entities to model are: \texttt{Post} and \texttt{Tag}.

\begin{lstlisting}[language=json]
{
  "name": "Post",
  "properties": {
    "title": { "type": "string" },
    "posted": { "type": "date" },
    "content": { "type": "text" },
    "permalink": { "type": "string" }
  }, 
  "relations": [{
    "name": "tags", 
    "type": "has_many", 
    "model": "Tag"
  }]
}
\end{lstlisting}

\begin{lstlisting}[language=json]
{
  "name": "Tag",
  "properties": {
    "name": { "type": "string" }
  }
}
\end{lstlisting}


\subsection{Pages}

\texttt{<page-posts>} show the list of posts.

It use \texttt{<api-collection-get>}, \texttt{<x-paginator>}, and \texttt{<x-list>} to retrieve, paginate and list the posts.

\begin{lstlisting}[language=HTML5]
<page-post>
</page-post>
\end{lstlisting}

\texttt{<page-post>} show a post. 

It is accessible via \texttt{<x-route path="/posts/:post\_id" page="page-post"}

\begin{lstlisting}[language=HTML5]
<page-post>
  <api-model-get name="Posts" 
    model-id="{{post_it}}" model="{{post}}">
  </api-model-get>
  <h1>{{post.title}}</h1>
  <h2>by {{post.author}}</h2>
  <h3>on {{post.date}}</h3>
  <div>{{post.content}}</div>
</page-post>
\end{lstlisting}
