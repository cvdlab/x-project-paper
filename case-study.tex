\section{Case study}
In this section we discuss the design and the implementation of a blog platform. 

\subsection{server-side}
For a blog platform the entities to model are: \texttt{Author}, \texttt{Post} and \texttt{Tag}.

\texttt{Post} model (defined below) represent a blog post.

\begin{lstlisting}[language=json]
{
  "name": "Post",
  "properties": {
    "title": { "type": "string" },
    "posted": { "type": "date" },
    "content": { "type": "text" },
  }, 
  "relations": [{ 
    "name": "author", 
    "type": "belongs_to",
    "model": "Author",
  }, {
    "name": "tags", 
    "type": "has_many",
    "model": "Tag",
  }]
}
\end{lstlisting}

\texttt{Tag} model represent a tag in a post.

\texttt{Author} model represent a blog author. It has properties that describe an author, such as \texttt{full\_name}, and one \texttt{has\_many} relation to \texttt{Post} model.

  
\subsection{client-side}
Client-side pages are encapsulated in elements that extend \texttt{<x-page>} element. These pages can be divided in two parts: Admin and User.

\subsubsection{Admin part}
The \emph{Admin part} is automatically generated. 
It consists of the following pages: \texttt{<page-collections>}, \texttt{<page-collection>}, \texttt{<page-model-edit>}.

\vspace{0.2cm}

\texttt{<page-collections>} is the main page. It show the collections of the app. In this case, these are \texttt{Authors}, \texttt{Posts} and \texttt{Tag}.

\vspace{0.2cm}

\texttt{<page-collection>} show the model instances of a collection.

\begin{lstlisting}[language=HTML5]
<page-collection>
  <api-collection-get 
    name="{{collection_name}}" 
    filter="{{filter}}"
    collection="{{collection}}">
  </api-collection-get>
  <part-collection-filter 
    name="{{collection_name}}"  
    filter="{{filter}}">
  </part-collection-filter>
  <part-list 
    list="{{collection}}">
  </part-list>
  <part-paginator 
    list="{{list}}" 
    filter="{{filter}}"
    current="{{page}}">
  </part-paginator>
</page-collection>
\end{lstlisting}

\vspace{0.2cm}

\texttt{<page-model-edit>} presents the forms to update a model.
The form is automatically generated from the model schema.
This page is composed by an \texttt{<api-model-get>} element that retrieve the model to edit. The model schema (retrieved with the model) is passed to a \texttt{<x-form>} element that presents an input element for each property of the model. The type of the input element corresponds to the type of the property (e.g. a \emph{boolean} property is editable via a \texttt{<x-checkbox>} element).

\subsubsection{User part}
The \emph{User part} must be designed by the developer.
It consists of the following pages: \texttt{<page-author>}, \texttt{<page-posts>} and \texttt{<page-post>}. 

\vspace{0.2cm}

\texttt{<page-posts>} show the list of posts. It use \texttt{<api-collection-get>}, \texttt{<x-paginator>}, and \texttt{<x-list>} to retrieve, paginate and list the posts.

\vspace{0.2cm}

\texttt{<page-post>} show a post. It is accessible via 
\texttt{<x-route path="/posts/:post\_id" page="page-post">}. It use \texttt{<api-model-get>} to retrieve the post using the \texttt{post\_id} parameter matched in the url.

\begin{lstlisting}[language=HTML5]
<page-post>
  <api-model-get name="Posts" 
    model-id="{{post_it}}" model="{{post}}">
  </api-model-get>
  <h1>{{post.title}}</h1>
  <h2>by {{post.author}}</h2>
  <h3>on {{post.date}}</h3>
  <div>{{post.content}}</div>
</page-post>
\end{lstlisting}
