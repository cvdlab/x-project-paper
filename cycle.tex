\section{Web development cycle}

We model the web development process as a four-steps procedure that can be
applyied recursively to each page (or view) of a web application as well as to
every single complex component (or widgets) of the page itself.

This modellization is based on the reasonable assumption that server side
operation on data models are nowadays be sufficently explored, and as proven by
the {\em Keystone.js} experience, at least one choice is available to automagically  
1) generate server-side CRUD methods on models with ACL capabilities and 
2) handle users and sessions, 
once a JSON description of data models and relations between them are provided to the
system. This very JSON descriptor documents drive the whole process, actually composed
by the following four steps.

% se avanza spazio si possono mettere dei nomignoli per ogni stepche potrebbero essere:
% 1. JSON data model description
% 2. Model actions definion
% 3. UI component definition
% 4. UI component assemplation

{\bf 1\textsuperscript{st} step - JSON data model description}. The JSON
descriptors must be defined, specifying data type, relation, and user role
read/write capabilities on particular portion of data.

{\bf 2\textsuperscript{nd} step - Model actions definion}. Since CRUD
operation coul not be enougth to describe all the needed operation further
actions on models can be defined and exposed via http verbs.

{\bf 3\textsuperscript{rd} step - UI component definition}. Then individual UI
component can be defined,  relying exclusively on CRUD operations and actions
available on data models.

{\bf 4\textsuperscript{th} step - UI component assemplation}. As last task,
previously defined UI component  have to be mounted to define application
views. Assembly should be kept as simple as possible, in the case of x-project
toolkit, it only consists of a juxtaposition of HTML5 tags.



% DA NOTARE CHE SIA I JSON DESCRIPTORS CHE LE COMPONENTI UI SINGOLE POSSONO ESSERE OLT CHE DEFINITE EX-NOVO,
% ANCHE RICERCATE IN UN DATABASE CHE NE METTE A DISPOSIZIONE PER MODELLI E COMPONENTI UI DI LARGO E/O FREQUENTE UTILIZZO.
