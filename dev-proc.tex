\section{Document-driven web development process}\label{sec:dev-proc}

The process to build a web application based on \brand{x-project} toolkit consists of the following four steps.

{\bf 1\textsuperscript{st} step - Models schemas definition}. A description of entities, properties, relations and data access policies are defined as JSON documents.

{\bf 2\textsuperscript{nd} step - HTTP RESTful API definition}. CRUD operations on models are automatically generated by the web framework (on the basis of input JSON documents) and further custom actions can be defined. All of them are exposed as HTTP RESTful API.

{\bf 3\textsuperscript{rd} step - UI components definition}. Distinct UI components can be defined, or retrieved from a collection of predefined components, configured and adapted. They represent the building blocks of the whole UI.

{\bf 4\textsuperscript{th} step - UI components assembly}. Distinct UI components are finally mounted to compose the application views. Assembly is kept as simple as possible: it only consists of a composition of HTML5 elements.

\vspace{0.2cm}

So the entire development process results driven by: 1) JSON documents describing entities of the application and 2) HTML template documents describing the UI components.

% This modellization is based on the reasonable assumption that server side
% operation on data models are nowadays be sufficently explored, and as proven by
% the {\em KeystoneJS} experience, at least one choice is available to automagically  
% 1) generate server-side CRUD methods on models with ACL capabilities and 
% 2) handle users and sessions, 
% once a JSON description of data models and relations between them are provided to the
% system. This very JSON descriptor documents drive the whole process, actually composed
% by the following four steps.

% se avanza spazio si possono mettere dei nomignoli per ogni stepche potrebbero essere:
% 1. JSON data model description
% 2. Model actions definion
% 3. UI component definition
% 4. UI component assemplation

% PARLARE DI DECOMPOSIZIONE VERTICALE??
