\section{Document-driven web development process}\label{sec:dev-proc}

In the following the activities performed to build an application are outlined to the aim of elicit a development approach that imposes a neat logic decomposition that strongly supports an engineered design of the web application. The identified main activities can be arranged in the following four sequential steps.

{\bf 1\textsuperscript{st} step - JSON documents definition}. The input JSON documents must be defined specifying data model schemas, model relations, user roles read/write capabilities on particular portion of data, and ancillary configurations (e.g. DBMS type). The output of this stage is a comprehensive set of HTTP RESTfull API to operate CRUD methods on defined data.

{\bf 2\textsuperscript{nd} step - Model actions definition}. Further actions on models besides CRUD ones must be defined in this step and exposed as RESTful API via HTTP verbs.

{\bf 3\textsuperscript{rd} step - UI component definition}. Distinct UI component must be defined, or retrieved from a collection of predefined components, configured and adapted. As concerns server communication, these components may avail only of the HTTP RESTfull API defined in the previous two steps.

{\bf 4\textsuperscript{th} step - UI component assembly}. Distinct UI components are finally mounted to compose the application views. Assembly is kept as simple as possible: it only consists of a composition of HTML5 elements.


% This modellization is based on the reasonable assumption that server side
% operation on data models are nowadays be sufficently explored, and as proven by
% the {\em KeystoneJS} experience, at least one choice is available to automagically  
% 1) generate server-side CRUD methods on models with ACL capabilities and 
% 2) handle users and sessions, 
% once a JSON description of data models and relations between them are provided to the
% system. This very JSON descriptor documents drive the whole process, actually composed
% by the following four steps.

% se avanza spazio si possono mettere dei nomignoli per ogni stepche potrebbero essere:
% 1. JSON data model description
% 2. Model actions definion
% 3. UI component definition
% 4. UI component assemplation

% PARLARE DI DECOMPOSIZIONE VERTICALE??
