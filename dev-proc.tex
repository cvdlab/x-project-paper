\section{Document-driven web development process}\label{sec:dev-proc}

The development process supported by the toolkit can be outlined as a cycle made by four steps that can be recursively applied to each ``part'' or component of the application. 

{\bf 1\textsuperscript{st} step - JSON document production}. The input JSON descriptors must be defined, specifying data model schemas, model relations, user roles read/write capabilities on particular portion of data, and ancillary configurations (e.g. DBMS type). The output of this stage is a comprehensive set of HTTP RESTfull API to operate CRUD methods on define data.

{\bf 2\textsuperscript{nd} step - Model actions definion}. Further actions on models besides CRUD ones must be defined in this step and exposed as RESTful API via HTTP verbs.

{\bf 3\textsuperscript{rd} step - UI component definition}. Then individual UI component must be defined here, relying exclusively on CRUD operations and actions available server-side on data models.

{\bf 4\textsuperscript{th} step - UI component assemblation}. Previously defined UI component are finally mounted to compose the application views. Assembly should be kept as simple as possible, in the case of \brand{x-project} toolkit, it only consists of a composition of HTML5 elements.


% This modellization is based on the reasonable assumption that server side
% operation on data models are nowadays be sufficently explored, and as proven by
% the {\em KeystoneJS} experience, at least one choice is available to automagically  
% 1) generate server-side CRUD methods on models with ACL capabilities and 
% 2) handle users and sessions, 
% once a JSON description of data models and relations between them are provided to the
% system. This very JSON descriptor documents drive the whole process, actually composed
% by the following four steps.

% se avanza spazio si possono mettere dei nomignoli per ogni stepche potrebbero essere:
% 1. JSON data model description
% 2. Model actions definion
% 3. UI component definition
% 4. UI component assemplation

Each application ``part'', outcome of the described cycle, is here intended as a vertical section or functional decomposition of the web application. 
For example... METTERE UN ESEMPIO DI DECOMPOSIZIONE VERTICALE

This approach impose a neat logic decomposition that strongly supports an engineeared design of the web application.
