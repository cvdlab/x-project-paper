\section{Introduction}
Lorem ipsum dolor sit amet, consectetur adipisicing elit, sed do eiusmod
tempor incididunt ut labore et dolore magna aliqua. Ut enim ad minim veniam,
quis nostrud exercitation ullamco laboris nisi ut aliquip ex ea commodo
consequat. Duis aute irure dolor in reprehenderit in voluptate velit esse
cillum dolore eu fugiat nulla pariatur. Excepteur sint occaecat cupidatat non
proident, sunt in culpa qui officia deserunt mollit anim id est laborum.


Section 2 presents the context where \brand{x-project} is positioned, overviewing the CMS evolution; section 3 introduces \brand{x-project}, its philosophy and architecture. Section 4 shows a case study in which is implemented a blog using x-project toolkit. Finally, section 5 draws conclusions.


One generally accepted definition of Content Management System is: a system that lets you apply management principles to content.

It is simple to elicit an evolutionary path in the history of management systems whose milestones are identifiable in \emph{Joomla!}, \emph{Wordpress} and \emph{KeystoneJS}.

Alongside these milestones entire constellations of analogous experiences popped up, but we considered them not relevant since they borrowed main features and constitutive approaches from cited ones. 

Starting from \emph{Joomla!}, a framework that drove in the engineering into the world of web content management. Joomla! powers more than 2,7\% of the largest 1,000,000 web sites in the world \cite{usage-cms}. Anyway, nowadays, \emph{Joomla!} results unwieldy and, due to its monolithic approach, not complied to current web features.

\emph{Wordpress}, instead is used by more than 23.3\% of the top 10 million websites (as of January 2015) \cite{usage-cms}. Wordpress develop CMS’s idea, driving in the intention to use CMSs to build Web Application. Wordpress, with its plugin, aims to limber user experience. The availability of more than 37,000 plugins, because it lets to create sites to non-experts too.
Anyway, further customizations, other than the ones introduced by plugins, are difficult to deploy due to loosely code engeneering of this tool.

Finally, \emph{KeystoneJs} stand in the last position. Minimal and agile, KeystoneJS, embody the new era of the CMS, letting the user to make his own personal web application.

In the spite of focusing on very minimal features, a mention goes to Ghost, a non-profit, open source blogging Node.js based platform, which represents a transversal experience, although minimal brings a new light way to blogging.

Eventually, \brand{x-project}, for the reasons the will be clear in the remaining of this paper, could be thought as the last standing in this path of decreasing monolithic approach and increasing customization and code engineering.

The remainder of this document is organized as follows. In
Section~\ref{related-work} we provide an overview of the state of the art in
the field of indoor document standards and related applications.
Section~\ref{advances} is devoted to describe the advances introduced by the
novel cartographic document proposed, while section~\ref{hijson-syntax}
presents the document syntax. Section~\ref{hijson-toolkit} reports about the
toolkit specifically developed to handle the new document format. In
Section~\ref{hijson-web-framework} it is depicted the overall architecture and
the implementation of the web based application framework, which is in turn
used to achieve the objectives stated above. Section~\ref{case-study} presents
a case-study application of the document format discussed in this paper.
Finally, Section~\ref{conclusions} proposes some conclusive remarks and future
developments.








To speed up web application development, frameworks mostly rely on external configuration files and less on procedural code \cite{6859693}.
