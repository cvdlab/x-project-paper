\section{Introduction}\label{sec:introduction}

Since the beginning of Internet, the ability to create and publish content on the web has made the success of Content Management Systems. Products like \href{http://www.joomla.org/}{Joomla!} or \href{https://wordpress.org/}{WordPress}, born to handle simple website or blogs, are evolved to support web applications of any sort (from personal portfolio, to on-line newspaper or on-line shopping), running as of January 2015 more than 25\% of the top ten million websites \cite{usage-cms}. This evolution has been allowed by a plugin-based architecture, where each plugin is responsible to handle a functionality subset of the whole application, presenting the user with a simple accessible configuration and management interface.

The large number of available plugins covers most of the common and frequently required customizations, thus avoiding to write ad-hoc code. Nevertheless, the implmentation of specific functional characteristics inevitably require to intervene at code level.

When the effort required to add custom features to a CMS results too expensive, a web framework can be adopted instead. A web framework consists of a set of software facilities that aims to alleviate the overhead associate with common development activities. Web application coding effort, while eased by the web framework, is anyway rewarded with an increased level of extensibility and customizability of the resulting application.

The most desiderable features for a wab framework are: a) user management, b) session management, c) automatic generation of CRUD methods on data models exposed via HTTP RESTfull API, d) data access policies management. In order to effectively speed up web applications development, these facilities should be provided mostly reling on external configuration files and less on procedural code \cite{6859693}.

% The main contribution of the work presented in this paper consists of the definition of a document-drive web development process which realies on Loopback, an open source web framework, and is supported by a software toolkit, named \brand{x-project}, whose designing choices and implementation are also discussed. This approach extremize the concept of the reuse of code whose complessive readability, maintainability and extendibility result dramaticaly increased. As a case study, a prototype software has been relized, which aims to syntetize the ease of use of a traditional CMS with the customizability of a modern web framework.

In this paper a software toolkit named \brand{x-project} is introduced. It consists of a Web Component library which applied over a very powerful web framework, i.e. Loopback by Strongloop, realizes an ibrid prototypal tool which bring togheter the customizability of a modern web framework with the ease of use of traditional CMSs.

Furthermore, the introduction of this tool implicitly defines a document-driven development process that extremize the concept of the reuse of code whose complessive readability, maintainability and extendibility result dramatically increased.

%Each component, thought as the analogous of the plugins in the CMS realm, is vertical piece of the tecnology stack, raging from client-side to server-side, responsible to cover an entire aspect of the web application, encapsulating  all the details for that single aspect: from user experience to application logic.

The remainder of this document is organized as follows. Section~\ref{sec:architecture} is devoted to describe the architecture and the tecnology stack exposed by applications developed with the toolkit, while section~\ref{sec:toolkit} presents the toolkit itself. Section~\ref{sec:case-study} reports about a case-study application: it is shown how the toolkit can be used to build a blog. Finally section~\ref{sec:dev-proc} summarizes the development process implicitly defined by the \brand{x-project} toolkit.

