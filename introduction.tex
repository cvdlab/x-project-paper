\section{Introduction}\label{sec:introduction}

Since the beginning of Internet, the ability to create and publish content on the web has made the success of Content Management Systems. Products like \href{http://www.joomla.org/}{Joomla!} or \href{https://wordpress.org/}{WordPress}, born to handle simple websites or blogs, are evolved to support web applications of any sort (from personal portfolio, to on-line newspaper or on-line shopping), running as of January 2015 more than 25\% of the top ten million websites \cite{usage-cms}. This evolution has been allowed by a plug-in based architecture, where each plug-in is responsible to handle a functionality subset of the whole application, presenting the user through a simple accessible configuration and management interface.

The large number of available plug-ins covers most of the common and frequently required customizations, thus avoiding to write ad-hoc code. Nevertheless, the implementation of specific functional characteristics inevitably require to intervene at code level.

When the effort required to add custom features to a CMS results too expensive, a web framework can be adopted instead. A web framework consists of a set of software facilities that aims to alleviate the overhead associate with common development activities. Web application coding effort, while eased by the web framework, is anyway rewarded with an increased level of extensibility and customizability of the resulting application.

The most desirable features for a web framework are: a) user management, b) session management, c) automatic generation of CRUD methods on data models exposed via HTTP RESTfull API, d) data access policies management. In order to effectively speed up web applications development, these facilities should be provided relying mostly on external configuration files and less on procedural code \cite{6859693}.

In this paper a software toolkit named \brand{x-project} is introduced. It consists of a Web Component library that is applied over a very powerful web framework, i.e. Loopback by Strongloop, and realizes an hybrid prototypal tool which brings together the customizability of a modern web framework with the ease of use of traditional CMSs.

Furthermore, the use of the toolkit alongside the web framework implicitly defines a document-driven development process that polarizes the concept of reusing the code whose overall readability, maintainability and extendibility result dramatically increased.

The remainder of this document is organized as follows. Section~\ref{sec:architecture} is devoted to describe the architecture and the technology stack exposed by applications developed by the toolkit, while section~\ref{sec:toolkit} presents the toolkit itself. Section~\ref{sec:dev-proc} outlines the development process implicitly defined by the \brand{x-project} toolkit. Finally section~\ref{sec:case-study} reports about a case-study application: it is shown how the toolkit can be used to build a blog.

