\documentclass{sig-alternate}

\begin{document}
%
% --- Author Metadata here ---
\conferenceinfo{WOODSTOCK}{'97 El Paso, Texas USA}
%\CopyrightYear{2012} % Allows default copyright year (20XX) to be over-ridden - IF NEED BE.
%\crdata{0-12345-67-8/90/01}  % Allows default copyright data (0-89791-88-6/97/05) to be over-ridden - IF NEED BE.
% --- End of Author Metadata ---

\title{x-project: next-generation API-centric HTML5 Web Components based Web Application and CMS framework}

% \titlenote{A full version of this paper is available as
% \emph{Author's Guide to Preparing ACM SIG Proceedings Using
% \LaTeX$2_\epsilon$\ and BibTeX} at
% \texttt{www.acm.org/eaddress.htm}}}
%
% You need the command \numberofauthors to handle the 'placement
% and alignment' of the authors beneath the title.
%
% For aesthetic reasons, we recommend 'three authors at a time'
% i.e. three 'name/affiliation blocks' be placed beneath the title.
%
% NOTE: You are NOT restricted in how many 'rows' of
% "name/affiliations" may appear. We just ask that you restrict
% the number of 'columns' to three.
%
% Because of the available 'opening page real-estate'
% we ask you to refrain from putting more than six authors
% (two rows with three columns) beneath the article title.
% More than six makes the first-page appear very cluttered indeed.
%
% Use the \alignauthor commands to handle the names
% and affiliations for an 'aesthetic maximum' of six authors.
% Add names, affiliations, addresses for
% the seventh etc. author(s) as the argument for the
% \additionalauthors command.
% These 'additional authors' will be output/set for you
% without further effort on your part as the last section in
% the body of your article BEFORE References or any Appendices.

\numberofauthors{4} %  in this sample file, there are a *total*
% of EIGHT authors. SIX appear on the 'first-page' (for formatting
% reasons) and the remaining two appear in the \additionalauthors section.
%
\author{
% You can go ahead and credit any number of authors here,
% e.g. one 'row of three' or two rows (consisting of one row of three
% and a second row of one, two or three).
%
% The command \alignauthor (no curly braces needed) should
% precede each author name, affiliation/snail-mail address and
% e-mail address. Additionally, tag each line of
% affiliation/address with \affaddr, and tag the
% e-mail address with \email.
%
% 1st. author
\alignauthor
Andrea D'Amelio\\
  \affaddr{Universit\`a Roma Tre}\\
  \affaddr{Dipartimento di Ingegneria}\\
  \affaddr{Universit\`a Roma Tre}\\
  \affaddr{Rome, Italy}\\
  \email{damelio@dia.uniroma3.it}
% 2nd. author
\alignauthor
Enrico Marino\\
  \affaddr{Universit\`a Roma Tre}\\
  \affaddr{Dipartimento di Ingegneria}\\
  \affaddr{Universit\`a Roma Tre}\\
  \affaddr{Rome, Italy}\\
  \email{marino@dia.uniroma3.it}
\and
% 3rd. author
\alignauthor
Tiziano Sperati\\
  \affaddr{Universit\`a Roma Tre}\\
  \affaddr{Dipartimento di Ingegneria}\\
  \affaddr{Universit\`a Roma Tre}\\
  \affaddr{Rome, Italy}\\
  \email{sperati@dia.uniroma3.it}
% 4th. author
\alignauthor
Federico Spini\\
  \affaddr{Universit\`a Roma Tre}\\
  \affaddr{Dipartimento di Ingegneria}\\
  \affaddr{Universit\`a Roma Tre}\\
  \affaddr{Rome, Italy}\\
  \email{spini@dia.uniroma3.it}
}
% There's nothing stopping you putting the seventh, eighth, etc.
% author on the opening page (as the 'third row') but we ask,
% for aesthetic reasons that you place these 'additional authors'
% in the \additional authors block, viz.
% \additionalauthors{Additional authors: John Smith (The Th{\o}rv{\"a}ld Group,
% email: {\texttt{jsmith@affiliation.org}}) and Julius P.~Kumquat
% (The Kumquat Consortium, email: {\texttt{jpkumquat@consortium.net}}).}
% \date{30 July 1999}
% Just remember to make sure that the TOTAL number of authors
% is the number that will appear on the first page PLUS the
% number that will appear in the \additionalauthors section.

\maketitle
\begin{abstract}
This paper provides a sample of a \LaTeX\ document which conforms,
somewhat loosely, to the formatting guidelines for
ACM SIG Proceedings. It is an {\em alternate} style which produces
a {\em tighter-looking} paper and was designed in response to
concerns expressed, by authors, over page-budgets.
It complements the document \emph{Author's (Alternate) Guide to
Preparing ACM SIG Proceedings Using \LaTeX$2_\epsilon$\ and Bib\TeX}.
This source file has been written with the intention of being
compiled under \LaTeX$2_\epsilon$\ and BibTeX.

The developers have tried to include every imaginable sort
of ``bells and whistles", such as a subtitle, footnotes on
title, subtitle and authors, as well as in the text, and
every optional component (e.g. Acknowledgments, Additional
Authors, Appendices), not to mention examples of
equations, theorems, tables and figures.

To make best use of this sample document, run it through \LaTeX\
and BibTeX, and compare this source code with the printed
output produced by the dvi file. A compiled PDF version
is available on the web page to help you with the
`look and feel'.
\end{abstract}

% A category with the (minimum) three required fields
\category{H.4}{Information Systems Applications}{Miscellaneous}
%A category including the fourth, optional field follows...
\category{D.2.8}{Software Engineering}{Metrics}[complexity measures, performance measures]

\terms{Theory}

\keywords{Web Application, Web Platform, Single Page Application, Content Management System, Web Components, API}

\section{Introduction}

A web application framework is a software framework that is designed to support the development of dynamic web applications.
To speed up web application development, action based frameworks mostly rely on external configuration files
Regarding to numerous open source web frameworks, it becomes so difficult for developers to select a suitable framework for web applications development. Because of that, the purpose of this paper is to help web developers to easier choose the right framework for development of their web applications. The paper will analyze a few open source Java component based frameworks. Additionally, the paper will provide basic features of the analyzed frameworks as well as represent their most important characteristics. Also, in this paper the analyzed web frameworks will be compared and summarized [1, 2, 3].

\section{State of the art}

Web  è una collezione di standard all'avanguardia che ci fornisce la possibilità di creare widget, è totalmente riusabile e non interferisce con il funzionamento della pagina in caso di cambiamento dell'implementazione interna del componente.
Nello sviluppo di un widget è importante utilizzare abilmente Javascript e HTML per la visualizzazione dinamica dei contenuti e gli standard di Web Components sono orientati verso ciò. L'unico, fondamentale, problema relativo allo sviluppo di Web Components è dettato dal DOM. Il DOM interno a un widget non è incapsulato dal resto della pagina. Questa mancanza potrebbe portare a problemi di sovrascritture da parte della pagina su elementi all'interno del widget.
Open-source software (OSS) is computer software with its source code made available with a license in which the copyright holder provides the rights to study, change and distribute the software to anyone and for any purpose.[21] Open-source software is developed in a collaborative public manner. Open-source software is the most prominent example of open-source development and often compared to (technically defined) user-generated content or (legally defined) open-content movements.[22]

\subsection{Web Application and CMS platforms}

One universal definition of Content Management System is: ``A system that lets you apply management principles to content''[4].
WordPress is a free and open-source CMS based on PHP and MySQL. Features include a plugin architecture and a template system [5]. WordPress was used by more than 23.3% of the top 10 million websites as of January 2015 [6]. WordPress is the most popular blogging system in use on the Web, at more than 60 million websites [7, 8].
Keystone.js is a Node.js CMS and Web Application Platform. It is an open source framework for developing database-driven websites, applications and APIs in Node.js. It is built on Express and MongoDB [9].
The LoopBack framework is a set of Node.js modules that you can use independently or together.  
An application interacts with data sources through the LoopBack model API, available locally within Node.js, remotely over REST, and via native client APIs for iOS, Android, and HTML5. Using these APIs, apps can query databases, store data, upload files, send emails, create push notifications, register users, and perform other actions provided by data sources and services.
Clients can call LoopBack APIs directly using Strong Remoting, a pluggable transport layer that enables you to provide backend APIs over REST, WebSockets, and other transports.
The following diagram illustrates key LoopBack modules, how they are related, and their dependencies. [10]

\subsection{Single Page Applications}

A single-page application (SPA), is a web application or web site that fits on a single web page with the goal of providing a more fluid user experience akin to a desktop application. In an SPA, either all necessary code - HTML, JavaScript, and CSS - is retrieved with a single page load, or the appropriate resources are dynamically loaded and added to the page as necessary, usually in response to user actions. The page does not reload at any point in the process, nor does control transfer to another page, although modern web technologies (such as those included in the HTML5 pushState() API) can provide the perception and navigability of separate logical pages in the application. Interaction with the single page application often involves dynamic communication with the web server behind the scenes.

\section{Next generation technologies and frameworks}

\subsection{WebComponents}

Web Components are a collection of standards which are working their way through the W3C and landing in browsers as we speak. In a nutshell, they allow us to bundle markup and styles into custom HTML elements. What's truly amazing about these new elements is that they fully encapsulate all of their HTML and CSS. That means the styles that you write always render as you intended, and your HTML is safe from the prying eyes of external JavaScript. [11]

\begin{description}
\itemsep1pt\parskip0pt\parsep0pt
  \item[Custom Elements] This specification describes the method for enabling the author to define and use new types of DOM elements in a document.
  \item[HTML Imports] HTML Imports are a way to include and reuse HTML documents in other HTML documents.
  \item[Templates] This specification describes a method for declaring inert DOM subtrees in HTML and manipulating them to instantiate document fragments with identical contents.
  \item[Shadow DOM] This specification describes a method of establishing and maintaining functional boundaries between DOM trees and how these trees interact with each other within a document, thus enabling better functional encapsulation within the DOM.
\end{description}

\subsection{Web Components Frameworks}

Web Components usher in a new era of web development based on encapsulated and interoperable custom elements that extend HTML itself.
There are three major frameworks built atop these new standards: Polymer, X-Tag and Bosonic.
Moreover there are some Web Application Frameworks that are moving their philosophy towards Web Components such as AngularJS [101, 200, 201] and EmberJS [102].
Polymer makes it easier and faster to create elements, from a button to a complete application across desktop, mobile, and beyond. The Polymer core provides a thin layer of API on top of web components. It expresses Polymer's opinion, provides the extra sugaring that all Polymer elements use, and is meant to help make developing web components much easier [103].
X-Tag allows to easily create elements to encapsulate common behavior or use existing custom elements to quickly get the behavior desired. X-Tag provides several powerful features that streamline element creation such as: custom events and delegation, mixins, accessors and component lifecycle functions [104].
Bosonic is a set of tools that enable the built and use reusable Web Components. By leveraging the power of DOM to build high-level elements, it allows to simplify application's code and benefits from 3rd-party elements [105].

\subsection{WebComponents libraries}

(customelements.io, comp kitchen, paper element)

\section{Contribution}

The real contribution of the work presented in this paper is...
la realizzazione di un API-centric CMS/SPA Application framework Web Components based.API-centrico basato su API builder strongloop loopback vantaggi immediati, separazione degli interessi, 

\section{X-Project}

X-Project is...

\subsection{Philosophy}

"Everything is an element", even a page. 
ogni parte del sito è incapsulata in un elemento, anche componenti funzionali come servizi http (chiamate ajax) possono essere incapsulate in un elemento...

\subsubsection{Components}

da Web services and web components
Component-based Software Engineering denotes the process of building software by (re)using pre-built software components. The main benefit of component-based software is the "time-to-market" [12], thus reducing the cost of developing the software. A software component is a unit of composition with contractually specified interfaces and explicit context dependencies. Additionally, a component is a software element that conforms to a component model and can be independently deployed and composed [13]. Component based software is developed interconnecting building blocks, therefore, a high degree of reusability and modularity is achieved by this type of applications [14].

\subsection{Architecture}

Client/Server API-centric architecture for Single Page Applications.

\subsubsection{Server-side}

\begin{description}
\itemsep1pt\parskip0pt\parsep0pt
  \item[Node.js web framework] 
    Node.js è un framework event-driven per il motore JavaScript V8, su piattaforme UNIX like. Si tratta quindi di un framework relativo all'utilizzo server-side di Javascript.
  \item[MongDB]
    MongoDB è un database non relazionale, orientato ai documenti. Classificato come un database di tipo NoSQL, MongoDB si allontana dalla struttura tradizionale basata su tabelle dei database relazionali in favore di documenti in stile JSON con schema dinamico (MongoDB chiama il formato BSON), rendendo l'integrazione di dati di alcuni tipi di applicazioni più facile e veloce.
\item[NoSQL]
    NoSQL (often interpreted as Not only SQL[1][2]) database provides a mechanism for storage and retrieval of data that is modeled in means other than the tabular relations used in relational databases. Motivations for this approach include simplicity of design, horizontal scaling, and finer control over availability. The data structures used by NoSQL databases (e.g. key-value, graph, or document) differ from those used in relational databases, making some operations faster in NoSQL and others faster in relational databases. The particular suitability of a given NoSQL database depends on the problem it must solve. 
  \item[Strongloop]
    StrongLoop offers a subscription-based product known as StrongLoop Suite. StrongLoop Suite includes three components: an open source private mobile Backend-as-a-Service mBaaS named LoopBack; a second component called StrongOps, which provides operations and real-time performance monitoring in a console; and a supported package of Node.js called StrongNode, containing advanced debugging, clustering and support for private npm registries.
\end{description}

\subsubsection{Client-side}

\begin{description}
\itemsep1pt\parskip0pt\parsep0pt
       \item[Polymer.js]
       \item[Web Components] 
\end{description}


\subsection{X-Elements components}

Come, polymer-project, offre una libreria di elementi, chiamata paper-elements, 
x-project offre una libreria di elementi, illustrati in questa sezione
possiamo classificare gli elementi in  strutturali, applicativi, form, di stile.



\subsubsection{Strctural elements}

elementi strutturali compongono la pagina

\begin{description}
\itemsep1pt\parskip0pt\parsep0pt
       \item[x-header], barra di navigazione
       \item[x-footer], per le informazioni a piè di pagina come autore, copyright, madeby, social buttons
       \item[x-drawer], struttura per menu laterale a scomparsa
       \item[x-page], struttura per una pagina
       \item[x-content], struttura per un contenuto generico
\end{description}





\subsubsection{Routing elements}

\begin{description}
\itemsep1pt\parskip0pt\parsep0pt
      \item[x-link], estensione dell'elemento anchor ({\tt <a>}) per evitare il comportamento di default e richidere la pagina al server, ma gestire il routing localmente 
\end{description}


\subsubsection{Form elements}

\begin{description}
\itemsep1pt\parskip0pt\parsep0pt
       \item[x-map] (google maps)
       \item[x-location] (google place API wrapper)
       \item[x-contact] (form di elementi di input aggregati per le informazioni di un contatto)
\end{description}



\subsubsection{Input elements}

\begin{description}
\itemsep1pt\parskip0pt\parsep0pt
       \item[x-input, x-textarea, x-number, x-date, x-datetime] elementi rispettivamente per testo breve, testo lungo, numeri, date, date e orario
\end{description}

\subsubsection{Style elements}

\begin{description}        
\itemsep1pt\parskip0pt\parsep0pt
       \item[x-bootstrap] e la libreria bootstrap adattata
per stilizzare elementi dello shadow dom nota su stilizzare gli elementi. lo
stile di un elemento si può mettere inline 
\end{description}

 

ma si possono anche stilizzare elementi dall'esterno, accedendo tramite i selettori CSS allo shadow dom, usando /deep/

link a github 


\subsection{Model definition}

To speed up web application development, action based frameworks mostly rely on external configuration files and less on Java code [15]
avviene tramite JSON


CMS definito tramite modelli. Un modello è definito da un json, estensione dei models di loopback. Il generatore di API di loopback usa questi modelli. Hanno un campo type. Nell'estensione, per non entrare in conflitto con loopback, viene aggiunto un campo {\tt \$type}.
A ogni tipo di dato è associato un campo di input corrispondente. 
Tali type sono: 

\begin{itemize}
\itemsep1pt\parskip0pt\parsep0pt
       \item number: x-number
       \item string: x-input,
       \item select: x-select,
       \item date: x-date,
       \item enum: x-radio,
       \item number: x-number,
       \item email: x-email,
       \item textarea: x-textarea,
       \item url: x-url,
       \item boolean: x-checkbox,
       \item money: x-money,
       \item password: x-password
\end{itemize}


\section{Conclusion}

The advantages of this approach ar

\subsection{Statistical comparison}

In the present time open source content management system (CMS) has gained a big market. Lots of varieties are available based on functionality and platform.
As told in there are different performance criteria like page load time (PLT), page size (PS), number of request, number of CSS and JS files etc. while comparing all these parameters we could come to conclusion that which CMS should be used under what conditions [19].

Generally all CMSs fulfill common task of content like create, edit, publish. But above mention CMS are providing good user support, security, more plug-ins, documentation etc than other.
 
As shown in [20] there are about thirty features to analyze to make a comparison of CMSs. We can design six important classes of features that summarize all the features.

Admin Management, Data Management, User Management, UI Management, Web Content Management, Multimedia Data Management.

\subsection{Security}

Third-party code inclusion is rampant, potentially exposing sensitive data to attackers. Protected Web components can keep private data safe from opportunistic attacks by hiding static data in the Document Object Model (DOM) and isolating sensitive interactive elements within a Web component.
The Web has evolved from including static images and document links to comprising Web applications with individual components provided by numerous service providers. When a Web application incorporates third-party components using remote scripts, the user's browser will run the third-party code within the security context of the Web application. This not only exposes the code's functionality to the Web application but also gives the included code full access to the Web application's client-side context, including the page's content, local data, and origin-protected functionality. 
This lack of code isolation can have severe consequences if the included code doesn't behave correctly.

Consequently, by including potentially untrusted remote scripts, a Web application developer accepts a certain risk, both for the site's integrity and for the safekeeping of user data [16].

Web components are the most viable starting point for creating a protection mechanism for private data and sensitive elements against opportunistic attackers. They offer the required flexibility to cope with the highly dynamic requirements of modern Web applications, as opposed to iframes, and already possess the capability to host a separate DOM tree using the shadow DOM, a property that is hard to achieve using JavaScript sandboxing technologies [17].

To leverage Web components to create protected Web components, we must be able to hide static data in the DOM tree, without it being accessible to opportunistic attackers. Second, protected Web components should be able to host interactive elements, without being vulnerable to script-based compromises-for example, through function-overriding or prototype-poisoning attacks [18].


\section{Future development}

estensione libreria elementi




%
% The following two commands are all you need in the
% initial runs of your .tex file to
% produce the bibliography for the citations in your paper.
\bibliographystyle{abbrv}
\bibliography{x-project-paper}  % doceng2015.bib is the name of the Bibliography in this case
% You must have a proper ".bib" file
%  and remember to run:
% latex bibtex latex latex
% to resolve all references
%
% ACM needs 'a single self-contained file'!
%
%\balancecolumns
\end{document}

